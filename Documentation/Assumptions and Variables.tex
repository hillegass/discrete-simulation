\documentclass{article}
\usepackage[utf8]{inputenc}
\usepackage{url}
\usepackage[margin=0.75in]{geometry}
\usepackage{float}
\usepackage{graphicx}

\setlength{\parskip}{0.7em}
\setlength{\parindent}{0em}

\begin{document}
	\begin{center}
    
    	% MAKE SURE YOU TAKE OUT THE SQUARE BRACKETS
		\LARGE{\textbf{CSE 6730, Group 37}} \\
        \vspace{1em}
        \Large{Description of parameters governing testing, natural recovery and transmission probability} \\
     
	\end{center}
    \begin{normalsize}
    

   	
   	\section{Assumptions}
   	\begin{itemize}
   	    \item Only unprotected acts modeled in this analysis
   	    \item No age mixing input preference
   	    \item Partner notification is stratified by sex and age, however in the absence of data on changes in this prevention strategy, the parameters are kept time invariant
   	    \item Only heterosexual partnerships.
   	    \item Treatment ensued immediately following identification of infection, although this may not always happen in practice.
   	\end{itemize}\\
    

	
	\begin{table}[H]
	\centering
    	\begin{tabular}{ |p{5cm}|p{7cm}|p{5cm}| } 
    		\hline
    		Parameter/Variable & Description & Distribution  \\ 
    		\hline
    		Population size & Population size for each age group & Uniformly distributed\\
    		Time step &	Time step implemented in the model & 	A day \\
			High risk & Fraction of the population defined as high risk	& 10\% (Assumption) \\
			Low risk & Fraction of the population defined as low risk & 90\% (Assumption)\\
			\hline
			\multicolumn{3}{|c|}{Testing symptomatic individuals} \\
			\hline
			Women &	Testing of symptomatic  women	& 1/(52*(0.079+0.072*Beta(4,4)))\\
			Men	& Testing of symptomatic men	& 1/(52*(0.079+0.072*Beta(4,4)))\\
			\hline
			
			\multicolumn{3}{|c|}{Casual partners} \\
			\hline
			High risk(HR)& Single, 65-79 HR	& Beta(3,60)\\
 						 & Single, 80-95 HR	& Beta(3,400)\\
			Low risk(LR)	 & Single, 65-79 LR	& Beta(1,160) \\
 						 & Single, 80-95 LR	&Beta(1,160)\\
 			\hline
 			\multicolumn{3}{|c|}{Among paired} \\
 			\hline
			High risk(HR)& Single, 65-79 HR	& Beta(10,70)\\
			Low risk(LR) & Single, 80-95 LR	& Beta(10,100)\\
			\hline
			
		\multicolumn{3}{|c|}{Transmission} \\
			\hline
			Transmission probability &Per act probability & Beta(5.5, 50)))\\
			With condom protection & condom effect parameter estimate is 1.6&Beta(5.5,50)^{1.6}\\
    		\hline
    		
		\multicolumn{3}{|c|}{Natural recovery} \\
			\hline
			Women & & 1/(52*(1.13+0.5*Beta(4,4.969)))\\
			Men & & 1/(52*(1.13+0.5*Beta(4,4.969)))\\

    		\hline
    	
    	\multicolumn{3}{|c|}{Treatment Success} \\
    	    \hline
			Efficiency of antibiotics & & Beta(190,8)))\\
    		\hline

    	\multicolumn{3}{|c|}{Partner Notification} \\
    	    \hline
			Women&Age65-79&Beta(4,3)\\
			&Age80-95&Beta(4,3)
			\\
			Men&Age65-79&Beta(4,3)\\
			&Age80-95&Beta(4,3)\\
    		\hline
        	\multicolumn{3}{|c|}{Condom Use} \\
    	    \hline
			Casual partners&Weighted prevalence & 0.131\\
			Paired & Weighted prevalence & 0.368
			\\
    		\hline
    		
	
        	
    		   		
    		
% https://www.ncbi.nlm.nih.gov/pmc/articles/PMC5477642/

    		
    	\end{tabular}
    	\label{tab:parameter}
    	\caption{Description of parameters governing testing, natural recovery and transmission probability}
   \end{table}
   
   	we chose to fix the fraction of the population defined as high risk at constant 10\%, but accommodate uncertainty in levels of risk behavior by varying the partner change rates by relationship states and age, in each of the risk groups. Defining a set proportion of the population to belong to a risk group and varying partner change rates is a modeling convention

    

	


\end{normalsize}
  
\end{document}
