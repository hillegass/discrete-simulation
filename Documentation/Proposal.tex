\documentclass{article}
\usepackage[utf8]{inputenc}
\usepackage[margin=0.75in]{geometry}

\begin{document}
	\begin{center}
    
    	% MAKE SURE YOU TAKE OUT THE SQUARE BRACKETS
		\LARGE{\textbf{CSE 6730, Group 37 Proposal}} \\
        \vspace{1em}
        \Large{Discrete Event Simulation} \\
     
	\end{center}
    \begin{normalsize}
    
    	\section{Project Title:}
        
Simulation of Disease Progression of Sexually Transmitted Diseases at Home Care Community.
      
		\section{Team Members:}
        
      \begin{enumerate}
      	\item Aiswarya Bhagavatula (GTID 903540374)
      	\item D. Aaron Hillegass (GTID 901988533)
      	\item Siawpeng Er (GTID 903413430)
      	\item Xiaotong Mu (GTID 903529807)
      \end{enumerate}
        
	   	\section{Problem Description and Purpose:}
        
    Sexually transmitted diseases (STD) is one of the problems troubling home care community. According to Athena Health, patients over 60 account for the biggest increase of in-office treatments for sexually transmitted infections.\\
    

    
    There are several factors that have led to the spread of STIs among older people:
    \begin{itemize}
	\item Lack of safer sex practices (such as condom use) in older individuals. People who became sexually active before AIDS are less likely to follow safe sex practices.
    \item Imbalance between the number of men and women. In retirement homes, there are typically significantly more women than men. It would not be surprising to find that the few healthy men would act as a nexus for sexually transmitted infections.
    \item Shame around testing and treatment. Older people (especially married older people) might be reluctant to tell their doctor about symptoms, get tested, and pursue treatment.
    \item Number of opportunities for transmission. In earlier times, we could expect sexual activity to diminish in the aging population.  However, with people living longer, healthier lives and the proliferation of safe erectile dysfunction drugs, people in retirement communities are more sexually active than their parents were at the same age -- especially if they live in close community with a large number of potential partners.
    \end{itemize}
    
    Discrete Event Simulation (DES) was been used for long time in many healthcare simulation, ranging from health care system operation, disease progression modeling, screening modeling and health behavior modeling \cite{lebcir2017, Zhang2018}. 
    
    
     A realistic simulation of the transmission of STIs in retirement homes could be useful in deciding between different interventions.  For example, would increasing condom use by 20\% be more effective than annual STI tests?
        
    \section{Methodology}
    We shall get the data from the Centers for Disease Control and Prevention (CDC) website for the related information on the related STDs. This includes:
    \begin{itemize}
    \item Rates in the general population at the ages at which people would enter retirement homes
   	\item Likelihood of transmission for different types of sexual activity (intercourse, oral, anal).
   	\item Time after infection before symptoms appear.
   	\end{itemize}
   	
   	We will also use a local retirement community to be modeled. From that administration we will find out:
   	\begin{itemize}
    \item Number of men and women 
    \item Ages at which people enter the community
    \item Duration that people stay in the community
    \item What, if any, STI testing and treatment are provided to the residents
   	\end{itemize}     
   	
   	Finally, we will do some interviews with residents to create a model of the individual:
   	\begin{itemize}
    \item Number of sexual partners per year
    \item History of STI testing and treatment 
    \item Marital status
    \item Gender
    \item Age
    \item Types of sexual activity that they engage in (if possible)
   	\end{itemize}     
   	
    
    \section{Development Platform}
    The programming language is Python $3$. Depends on the suitability of the project, we plan to provide a Jupyter notebook for user interaction, or just a command line execution.
    
    \section{Data source}
    We plan to obtain our data from CDC. The data is used for the projection of the disease progression. Subsequently, we shall use the data for designing the interaction within population in our simulated home care. 
    
    \section{Division of Labor}
    As we move forward on our project, we plan to work concurrently. The timeline is as below:
       
    \begin{center}
    	\begin{tabular}{ |c|c|c| } 
    		\hline
    		Task & Duration  \\ 
    		\hline
    		Data collection & 2 weeks \\ 
    		Modeling design and implementaion & 4 weeks \\ 
    		Modeling revised & 4 weeks \\ 
    		\hline
    	\end{tabular}
    \end{center}
    
    
    
    \bibliographystyle{plain}
    \bibliography{reference}
\end{normalsize}
  
\end{document}
