\documentclass{article}
\usepackage[utf8]{inputenc}

\usepackage[margin=0.75in]{geometry}

\setlength{\parskip}{0.7em}
\setlength{\parindent}{0em}

\begin{document}
	\begin{center}
    
    	% MAKE SURE YOU TAKE OUT THE SQUARE BRACKETS
		\LARGE{\textbf{CSE 6730, Group 37 Proposal}} \\
        \vspace{1em}
        \Large{Discrete Event Simulation} \\
     
	\end{center}
    \begin{normalsize}
    
    	\section{Project Title}
        
Simulation of the Spread of Syphilis within Group Housing for the Elderly
      
		\section{Team Members}
        
      \begin{enumerate}
      	\item Aiswarya Bhagavatula (GTID 903540374)
      	\item D. Aaron Hillegass (GTID 901988533)
      	\item Siawpeng Er (GTID 903413430)
      	\item Xiaotong Mu (GTID 903529807)
      \end{enumerate}
        
	   	\section{Problem Description and Purpose}
        
    More and more communities of the elderly are suffering from outbreaks of sexually transmitted infections. According to Athena Health, patients over 60 account for the biggest increase of in-office treatments for sexually transmitted infections.
    
    For this study, we are going to focus on syphilis, but the methodology and resulting simulation could easily be applied to other treatable, non-deadly STIs like chlamydia and gonorrhea.
    
    There are several factors that have led to the spread of STIs among older people (especially in group housing):
    \begin{itemize}
	\item Lack of safer sex practices (such as condom use) in older individuals. People who became sexually active before AIDS are less likely to follow safe sex practices.
    \item Imbalances between the number of men and women. In retirement homes, there are typically significantly more women than men. It would not be surprising to find that the few healthy men would act as a nexus for sexually transmitted infections.
    \item Shame around testing and treatment. Older people (especially married older people) might be reluctant to tell their doctor about symptoms, get tested, and pursue treatment.
    \item Number of opportunities for transmission. In earlier times, we could expect sexual activity to diminish in the aging population.  However, with people living longer, healthier lives and the proliferation of safe erectile dysfunction drugs, people in retirement communities are more sexually active than their parents were at the same age -- especially if they live in close community with a large number of potential partners.
    \item Antibiotic resistance. Old people living in community are likely to get other kinds of bacterial infections, like strep, and take antibiotics. In the past, this was likely to wipe out undiagnosed syphilis (or chlamydia or gonorrhea) as a side-effect. As these STIs have evolved to become more antibiotic resistant, a strep-sized dose of amoxicillin is less likely to do the job.
    \end{itemize}
    
    Discrete Event Simulation (DES) was been used for long time in many healthcare simulation, ranging from health care system operation, disease progression modeling, screening modeling and health behavior modeling \cite{lebcir2017, Zhang2018}. 
    
    
     A realistic simulation of the transmission of STIs in retirement homes could be useful in deciding between different interventions.  For example, would increasing condom use by 20\% be more effective than annual STI tests?
        
     \section{Data source}
    We will get the data from the Centers for Disease Control and Prevention (CDC) website for parameters on syphilis. This includes:
    \begin{itemize}
    \item Rates in the general population at the ages at which people would enter retirement homes
   	\item Likelihood of transmission for different types of sexual activity (intercourse, oral, anal).
   	\item Time after infection before symptoms appear.
   	\end{itemize}
   	
   	We will also use a local retirement community to be modeled. From that administration we will find out:
   	\begin{itemize}
    \item Number of men and women 
    \item Ages at which people enter the community
    \item Duration that people stay in the community
    \item What, if any, STI testing and treatment are provided to the residents
   	\end{itemize}     
   	
   	Finally, we will do some interviews with residents to create a model of the individual:
   	\begin{itemize}
    \item Number of sexual partners per year
    \item History of STI testing and treatment 
    \item Marital status
    \item Gender
    \item Age
    \item Types of sexual activity that they engage in (if possible)
   	\end{itemize}     
   	
   	\section{Methodology}
   	Our simulation will first simulate a population of people entering and exiting a single retirement community. It will use stochastic methods to give them an initial age, gender, and infection status.  It will also remove these people as the get older and move somewhere else or die. When someone dies or moves away, 
   	
   	Within that population, we will update each individual's infection status as they become infected and get treated. We will also track if they have become symptomatic. Thus, each time an uninfected person engages in sexual activity with an infected person, we will roll the dice to decide if the uninfected person becomes infected. Each person will be symptomatic for some amount of time before seeking testing and treatment.
   	
   	We will test different interventions:
   	\begin{itemize}
    \item Increasing condom usage
    \item Periodic testing and treatment of the whole community 
    \item Promoting monogamous fluid bonding
	\item Working toward equal numbers of men and women in the community
   	\end{itemize}     
    
    \section{Development Platform}
    The programming language is Python $3$. Depends on the suitability of the project, we plan to provide a Jupyter notebook for user interaction, or just a command line execution.
    
    We will use heapq as our priority queue.
    
    We will use Matplotlib to do our data visualization.
    
    \section{Division of Labor}
    As we move forward on our project, we plan to work concurrently. The timeline is as below:
       
    \begin{center}
    	\begin{tabular}{ |c|c|c| } 
    		\hline
    		Task & Duration  \\ 
    		\hline
    		Data collection & 2 weeks \\ 
    		Modeling design and implementaion & 4 weeks \\ 
    		Modeling revised & 4 weeks \\ 
    		\hline
    	\end{tabular}
    \end{center}
    
    
    
    \bibliographystyle{plain}
    \bibliography{reference}
\end{normalsize}
  
\end{document}
